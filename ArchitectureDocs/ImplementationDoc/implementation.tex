We originally decided to use the the architectures: ModelView Controller(MVC), Service Oriented Architecture(SOA), Observer Patterns(OP) and State Machine.


\subsection{Model View Controller}
MVC is the architecture that we have focused the most on. This makes it easy for the developers to get an overview of the project.

Models contains all the data in the game. This consists of game data, character data, character-, cursor- and map data. As we areusing the MVC architecture, these classes only contain raw data, nothing to view them at all.

Views contains everything that can be displayed. The different views then get their data from the models and display them accordingly. Defined as views, we have sprites, images, text viewing and the menu view.

The controllers handles all the data from the models handles all updates and uses the views to display the result.\\


\begin{tabular}{ll}
Models & \\
&Box\\
&Character\\
&CursorModel\\
&GameModel\\
&GameOptions\\
&Map\\
&MenuEntry\\
&Move\\
&Player\\
&PowerUp\\
&Weapon\\

Views & \\
&ImageTexture\\
&IView\\
&MenuView\\
&Sprite\\
&TextBox\\

Controllers & \\
&CameraController\\
&CharacterController\\
&CursorController\\
&GamepadController\\
&GamePlayController\\
&IController\\
&MapController\\
&MenuController\\
&OverlayMenuController\\
&SoundController\\
\end{tabular}



\subsection{Service Oriented Architecture}
We found that this was a nice addition to MVC. We have implemented a class, ControllerViewManager which acts as an interface for the controllers and views. This class then handles all of the world updating as well as the rendering of the game and handling of sprites. 

By using this, we dont need to worry about adding or removing components and makes it very easy to add more functionality to the game, as it then just has to be added to the SOA interface.


\subsection{Observer pattern}



\subsection{State Machine}